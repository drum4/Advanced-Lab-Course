% !TeX root = f44_shortreport_schmitt_kleinbek.tex
\begin{abstract}
%\textbf{\Large -\abstractname-}
\paragraph{Abstract}
\itshape
The Zeeman effect is an atomic physics phenomen that describes how spectral lines of an element are split when the magnetic moment of the atom is coupled to an external magnetic field.
The aim of this experiment was to observe the normal Zeeman effect in cadmium and then to investigate the splitting of the spectral lines as a function of the magnetic field strength.
In a second part of the experiment we determine the wavelength of the red cadmium line by using a Czerny Turner spectrometer.
We also had to determine the spectral line of an unknown element, which unfortunately was not possible because we could not resolve it. 
In our measurement we determined only the wavelength $\lambda_\text{Cd} = \SI{643.8(29)}{\nano\meter}$.
In addition, the Bohr magneton $\mu_\text{B}$ could be calculated from both test parts, for which we obtained the value $\mu_\text{B} = \SI{10.3(5)e-24}{\joule\per\tesla}$.
\end{abstract}