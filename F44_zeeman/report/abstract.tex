% !TeX root = f44_shortreport_schmitt_kleinbek.tex
\begin{abstract}
The Zeeman effect is an atomic physics phenomenon that describes how spectral lines of an element are split when the magnetic moment of the atom is coupled to an external magnetic field.
The aim of this experiment was to observe the normal Zeeman effect in cadmium and then to investigate the splitting of the spectral lines as a function of the magnetic field strength.
In a second part of the experiment we determine the wavelength of the red cadmium line by using a Czerny Turner spectrometer.
In our measurement we determined the wavelength $\lambda_{Cd} = \SI{643.8(29)}{\nano\meter}$.
In addition, the bohr magneton $\mu_B$ could be calculated from both test parts, for which we obtained $\mu_B = \SI{10.3(5)e-24}{\joule\per\tesla}$.
\end{abstract}