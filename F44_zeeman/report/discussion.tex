% !TeX root = f44_shortreport_schmitt_kleinbek.tex
\section{Critical Comment}
The aim of the experiment was to determine the Bohr magneton using the Zeeman effect and then to determine the wavelengths of the red cadmium line and an unknown element using a Czerny-Turner spectrometer.\\
In the first part of the experiment, the magnet used was examined for hysteresis effects which, however, turned out to be negligible.
The external errors, such as the manual insertion of the Hall probe, probably outweighed the corresponding measurements.
Next, the Zeeman effect itself was observed qualitatively with a cadmium lamp.
The polarization and the number of interference lines in longitudinal and transverse direction were investigated.
The observations were consistent with the previous theoretical considerations.
From the transverse direction of observation we could additionally obtain information about the line position and thus calculate Bohr's magneton.
We have obtained an experimental value of $\mu_{B} = \SI{9. 7(8)e24}{\joule\per\tesla}$ that is not significantly different from the literature value $\mu_{\text{B}_\text{lit}} = \SI{9. 274e-24}{\joule\per\tesla}$.
Remarkable here is the large error of over 8\% for the average of 3 measurements.
The error results mainly from the inaccuracy of the fit curves, since the standard deviation was used as the error of the position of the peaks.\\
In the second part of the experiment, the wavelengths of cadmium and another element were to be determined. For the strong red cadmium line a wavelength of $\lambda_\text{Cd}= \SI{643.8(29)}{\nano\meter}$ was obtained, which did not differ significantly from the literature value $\lambda_{\text{Cd}_\text{lit}} = \SI{643.8}{\nano\meter}$.
Unfortunately, we could not see a line in the spectrum for the unknown element.\\
Overall, the test was therefore satisfactory, as all measured values in the error range corresponded to the literature values.
As a possible improvement, we would suggest an optical bench for\\mounting the optical elements, which would reduce the number of inadvertent misalignments.
On the other hand, mechanical mounting would be advantageous for the Hall probe, as the hysteresis effect would obviously be very small and thus a source of error could be eliminated.