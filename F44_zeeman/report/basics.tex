% !TeX root = f44_shortreport_schmitt_kleinbek.tex
\section{Theoretical Basics}
\subsection{Normal Zeeman effect}
To understand the basics of the Zeeman effect, we first assume that the magnetic moment of an electron $I$ can be sufficiently described by Bohr's atomic model.\\
According to this model, the electron orbits the atomic nucleus as point mass $m_e$ with velocity $v$ and charge $e$ at a distance $r_\text{B}$, the Bohr radius.
Using this approximation, the orbital magnetic moment
\begin{align}
\vec{\mu}_l = \frac{evr}{2}\cdot\vec{n}
\end{align}
is obtained.
\begin{figure}[ht]
\centering
% !TeX root = ..//f44_shortreport_schmitt_kleinbek.tex
\begin{tikzpicture}[scale=.75]
\fill (0,0) circle (4pt);
\draw [thick](0,0) ellipse (3cm and 1cm);

\fill (2,-0.73) circle (2.5pt) node[below]{$e$};
\draw[arrows=-{Stealth[scale=1]}, thick] (2,-.73)--+(.75,.2);

\draw[arrows=-{Stealth[scale=1]}, thick] (0,0)--+(0,3) node [below right]{$\pmb{L}$};

\draw[thick] (0,0) -- (2,0.73) node [midway, below right]{$\pmb{r}$};

\draw[thick, dashed] (0,0) -- (0,-1);
\draw[arrows=-{Stealth[scale=1]}, thick] (0,-1)--+(0,-2) node [midway,right]{$\pmb{\mu}$};
\end{tikzpicture}
\caption{Bohr model of an electron with the angular momentum $\vec{L}$ and the magnetic moment $\vec{\mu}$}
\label{fig:bohrangular}
\end{figure}
Thus $\vec{n}$ is the normal vector, perpendicular to the disk on which the electron moves (see Fig. \ref{fig:bohrangular}).\\
You can easily see that the magnetic moment resembles the angular momentum of the electron
\begin{align}
\vec{l} = \vec{r} \times \vec{p} = m_e r v \cdot \vec{n}.
\end{align}
If an external magnetic field $\vec{B}$ is applied, it interacts with the magnetic moment of the electron so that the energy level
\begin{align}
\Delta E_\text{pot} = -\vec{\mu}_l \cdot \vec{B} = \frac{e}{2m_e} \cdot \vec{l} \cdot \vec{B}
\end{align}
changes.\\
Now the angular momentum of the electron along the magnetic field vector $\vec{B}$ is quantized in the form
\begin{align}
\abs{\vec{l}} = \sqrt{l(l+1)}\hbar
\end{align}
with the quantum number $l = 0, 1,…, n-1$ and the $z$-component
\begin{align}
l_z = m_l \hbar,
\end{align}
which runs in the range $-l \leq m_l \leq l$.\\
This allows you to simplify the energy difference to
\begin{align}
\Delta E_\text{pot} = \frac{e \hbar}{2 m_e} m_l B = \mu_\text{B} m_l B,
\end{align}
where $\mu_B$ describes the Bohr magneton we are looking for.\\
Changing the energy level by $\Delta E_\text{pot}$ causes the original energy level with angular momentum $l$ to split into $2l+1$ lower levels with the same angular momentum $l$ but different $m_l$ (see Fig. \ref{fig:splitting}).\\

\begin{figure}[ht]
\centering
% !TeX root = ..//f44_shortreport_schmitt_kleinbek.tex
\begin{tikzpicture}[scale=.7]
\draw[level] (0,0) node[left] {p} -- node[above] {$l=1$} (2,0);
\draw[level] (0,3) node[left] {d} -- node[above] {$l=2$} node[yshift=1.5cm] {$\vec{B}=0$} (2,3);

\draw[connect] (2,0) -- (3,-.5) (2,0) -- (3,0) (2,0) -- (3,.5);
\draw[level] (3,-.5) -- (8,-.5) node[right] {\scriptsize $m_l=-1$};
\draw[level] (3,0) -- (8,0) node[right] {\scriptsize $m_l=0$};
\draw[level] (3,.5) -- (8,.5) node[right] {\scriptsize $m_l=1$};

\draw[connect] (2,3) -- (3,2) (2,3) -- (3,2.5) (2,3) -- (3,3) (2,3) -- (3,3.5) (2,3) -- (3,4);
\draw[level] (3,2) -- (8,2) node[right] {\scriptsize $m_l=-2$};
\draw[level] (3,2.5) -- (8,2.5) node[right] {\scriptsize $m_l=-1$};
\draw[level] (3,3) -- node[yshift=1.46cm] {$\vec{B}\neq0$} (8,3) node[right] {\scriptsize $m_l=0$};
\draw[level] (3,3.5) -- (8,3.5) node[right] {\scriptsize $m_l=1$};
\draw[level] (3,4) -- (8,4) node[right] {\scriptsize $m_l=2$};

\draw[arrows=-{Stealth[scale=1]}, color=red](3.5,3)--(3.5,.5);
\draw[arrows=-{Stealth[scale=1]}, color=red](3.9,2.5)--(3.9,0);
\draw[arrows=-{Stealth[scale=1]}, color=red](4.3,2)--(4.3,-.5);
\draw[] (3.5,-1.3) node {\scriptsize $\Delta m_l=1$};

\draw[arrows=-{Stealth[scale=1]}, color=red](5.2,3.5)--(5.2,.5);
\draw[arrows=-{Stealth[scale=1]}, color=red](5.6,3)--(5.6,0);
\draw[arrows=-{Stealth[scale=1]}, color=red](6,2.5)--(6,-.5);
\draw[] (5.4,-1.3) node {\scriptsize $\Delta m_l=0$};

\draw[arrows=-{Stealth[scale=1]}, color=red](6.7,4)--(6.7,.5);
\draw[arrows=-{Stealth[scale=1]}, color=red](7.1,3.5)--(7.1,0);
\draw[arrows=-{Stealth[scale=1]}, color=red](7.5,3)--(7.5,-.5);
\draw[] (7.5,-1.3) node {\scriptsize $\Delta m_l=-1$};

\end{tikzpicture}
\caption{Schematic of the energy levels trough the normal Zeeman effect}
\label{fig:splitting}
\end{figure}

Alternatively, this equation can also be derived quantum mechanically.
All spins and torques of the electrons of an atom are considered as individual sums, as well as the total spin $\vec{J}$ required for the Hamilton operator in the external magnetic field.\\
This gives the equation
\begin{align}
\Delta E_\text{pot} = \mu_\text{B} \cdot M_J \cdot B \cdot g_j
\end{align}
with the Landé factor $g_j$, which is one for the normal Zeeman effect we needed.
The matrix element $M_J$ that appears in the equation is explained again in the following section.

\subsection{Selection Rules and polarization of light}
When an electron \enquote{jumps} between two electron shells $E_i$ and $E_k$, it emits a photon with a wavelength $\lambda$, which depends on the energy difference of the electron shells
\begin{align}
\frac{hc}{\lambda} = E_\text{photon} = \Delta E = E_i - E_k.
\end{align}
However, there are restrictions on which transitions are possible.
Important for this is the dipole matrix element
\begin{align}
M_{ik} = e \int \psi_i^* \, \vec{r} \, \psi_k \dd{V},
\end{align}
which describes the transition probability between the electron shells $k$ to $i$.
It must also have at least one component other than zero for a transition from $k$ to $i$ to be possible.\\
If we evaluate the integral in three spatial directions, we get the conditions
\begin{align}
\Delta M_J &= M_{J,i} - M_{J,k} = 0, \pm 1,\\
\Delta L &= L_i - L_k = \pm 1,\\
\Delta S &= 0.
\end{align}
Assuming the magnetic field vector $\vec{B}$ points in the z-direction, only the matrix element $(M_{ik})_z$ cannot become zero for $\Delta M_J = 0$.
These transitions are called $\pi$ transitions.
These are equivalent to a dipole oscillating along the $z$-axis, which means that no radiation is emitted along the $z$-direction. In the other two directions, this oscillation of the dipole can be observed as linearly polarized light.\\
If $\Delta M_J = \pm 1$, we get $\sigma$ transitions, for which the $z$-component of the dipole matrix becomes zero and the $x$- and $y$-components are not equal to zero, but phase shifted to each other by $\nicefrac{\pi}{2}$.
This causes us to observe circularly polarized light along the $z$-axis and linearly polarized light along the $x$- and $y$-axis.\\

The direction of observation along the $z$-axis is called longitudinal and for $\Delta M_J = + 1$ and $\Delta M_J = - 1$ circular polarized $\sigma$-lines should be visible.\\
In contrast, linearly polarized $\pi$-lines should be observed in the transverse direction (perpendicular to the $z$-axis).