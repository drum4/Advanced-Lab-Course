% !TeX root = f44_shortreport_schmitt_kleinbek.tex
\section{Introduction}
The Zeeman effect was studied first in 1896 by the Dutch physicist Peter Zeeman when he observed the widening of the yellow D-lines of burning sodium between strong magnets.
Later he found out that the widening of the lines was actually a division in up to 15 components.\\
The spectral lines of an element arise when an electron emits a photon at the transition between different energy levels, whose wavelength depends on the energy difference between the levels.
If a strong external magnetic field is applied, individual energy levels are changed by coupling the magnetic moment of the electron with the external magnetic field, which leads to a splitting of the spectral lines.
A distinction is made between the normal Zeeman effect observed in the experiment and the anomalous Zeeman effect.
These differ in the total spin $\vec{S}$ of the electron, which is $\vec{S} = 0$ at the normal Zeeman effect and $\vec{S}\neq 0$ at the anomalous Zeeman effect.