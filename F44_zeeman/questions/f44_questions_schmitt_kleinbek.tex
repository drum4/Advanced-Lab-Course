\documentclass[12pt,
				 a4paper,
				%landscape,
				%headings=Big,
				 ]{scrartcl}
\usepackage{polyglossia}			%Sprache
\usepackage{12many}				%Mengen
\usepackage{amsmath}				%Mathe
\usepackage{amsopn}				%Operatoren
\usepackage{amssymb}				%Symbole
\usepackage{amsthm}				%Theorem
\usepackage{array}					%Tabelle
\usepackage{bbm}					%Mengensymbol
\usepackage[style=numeric,
			natbib=true,
			backend=biber,
			]{biblatex}				%Literatur
\usepackage{blindtext}				%Blindtext
\usepackage{booktabs}				%Tabellen
\usepackage{braket}				%Braket
\usepackage{colortbl}				%Farbentabelle
\usepackage[german=quotes,
			 autostyle=true,
			 ]{csquotes}			%Anführungszeichen
\usepackage{empheq}				%Gleichung
\usepackage{fancybox}				%Box
%\usepackage[Bjornstrup
	%		 ]{fncychap}			%Kapitellayout
\usepackage{graphicx}				%Graphik
\usepackage{marvosym}			%Symbole
\usepackage{listings}				%Code
\usepackage{marginnote}			%Kommentar
\usepackage{mathdots}				%Punkte
\usepackage{mathtools}			%Bugfix
\usepackage{microtype}			%Typographie
\usepackage{multirow}				%Zeilen
\usepackage{nicefrac}
\usepackage{pdfpages}				%Pdf
\usepackage{pgfplots}				%Diagramme
\pgfplotsset{compat=1.5}
\usepackage{physics}				%Physik
\usepackage{relsize}				%Größe
\usepackage{scrhack}				%Verbesserung
\usepackage[headsepline,
			 automark,
			 ]{scrlayer-scrpage}		%Kopfzeile
\usepackage{slashed}				%Gestrichen
\usepackage[separate-uncertainty,
			per-mode=symbol,
			]{siunitx}				%Einheiten
\usepackage{subcaption}			%Gleitumgebung
\usepackage{tcolorbox}				%Boxen
\usepackage{tikz}					%Zeichnen
\usepackage{upgreek}				%Aufrecht
\usepackage{xcolor}				%Farbe
\usepackage{xltxtra}				%Fontec

\setmainlanguage{english}
\setmainfont{Linux Libertine O}
\setsansfont{Linux Biolinum O}
\addbibresource{}

\titlehead{\begin{minipage}{.5\textwidth}
			\begin{flushleft}
			\includegraphics[scale=.065]{logo_color.jpg}
			\end{flushleft}
 	 		\end{minipage}
			\begin{minipage}{.5\textwidth}
			\begin{flushright}
			\includegraphics[scale=.8]{logo_fp.png}
			\end{flushright}
 	 		\end{minipage}}
\title{F44 Zeeman spectroscopy}
\subtitle{Questions to the experiment}
\author{Nils Schmitt \and Timo Kleinbek}
\publishers{Supervisor: Srinivas, Hemkumar}
\date{Conducted in August 2018}

%\automark{chapter}
\pagestyle{scrheadings}
\chead{}
\ohead{\headmark}
\ihead{}

\setcounter{tocdepth}{1}
\setcounter{secnumdepth}{15}


\usepackage[bookmarksopenlevel=section,
			linkcolor=blue,
			colorlinks=true,
			urlcolor=blue,
			citecolor=blue,
			 ]{hyperref}			%Verweise
			 
\begin{document}
\maketitle

\section*{Is it possible to observe the Sodium Zeeman splitting with current setup?
What needs to be changed if needed?}
For Sodium the energy of the excited electron is splitted into two levels.
The atomic electron transition to the ground state emits photons with the wavelengths
\begin{align*}
\lambda_1&=\SI{589.0}{\nano\meter},\\
\lambda_2&=\SI{589.6}{\nano\meter}.
\end{align*}
The free spectral range of both transitions follows the formula
\begin{align}
\Delta\lambda=\frac{\lambda^2}{2d\cdot\sqrt{n^2-1}}
\end{align}
with the thickness $d$ of the Lummer-Gehrcke-Plate and the refractive index $n$.\\
So we get the free spectral range
\begin{align*}
\Delta\lambda&=\SI{4.053e-11}{\meter}.
\end{align*}
In the best case the peaks of one transition are always in the middle of the free spectral range of the other.
To resolve lines in this range the Zeeman splitting is only allowed in the quarter of this.
Therefore we use the equation
\begin{align}
\delta\lambda&=\frac{hc}{h\frac{c}{\lambda}+\Delta E}-\lambda\\
\intertext{with the energy difference}
\Delta E&=\mu_\text{B}\cdot m_l\cdot B.
\end{align}
If we solve according to the current
\begin{align}
I=\frac{hc}{\mu_\text{B}\overline{a}}\left(\frac{1}{\delta\lambda+\lambda}-\frac{1}{\lambda}\right)
\end{align}
with the ratio $\overline{a}$ between the magnetic field and the current, we see that the current must be under $I=\SI{10}{\ampere}$.
But generally the peaks are not in the middle of their free spectral range so the resolution will be more poor.\\
There is one solution for example which is called an Ethalon.
This is using an Fabry-Perot interferometer where you can observe both lines seperatly.
It works with an resonator which only transmit light following the rule
\begin{align}
f=\frac{c}{2nL}\cdot m
\end{align}
with the length of the resonator $L$ and $m\in\mathbbm{C}$.\\
In contrast to the Lummer-Gehrcke-Plate the Fabry-Perot interferometer can change the length, so the different wavelengths can be observed alone.




\section*{Calculate the $\nicefrac{e}{m}$ ratio for an electron using the Bohr magneton value.}
Because the electron mass is difficult to determine directly, the charge-mass-ratio ($\nicefrac{e}{m_\text{e}}$) is determined from measurements with the elementary charge instead.\\
The 2014 CODATA recommended value is
\begin{align*}
\frac{e}{m_\text{e}}=\SI{-1.759e11}{\second\ampere\per\kilo\gram}.
\end{align*}
We can check this value with the help of our measurement for the Bohr magneton $\mu_\text{B}=\SI{9.7(8)e-24}{\joule\per\tesla}$ and test how good the Zeeman effect is for determining the electron mass.\\
For this we use the already known equation
\begin{align}
\mu_\text{B}&=\frac{\hbar e}{2m_\text{e}}\\
\intertext{and solve according to the charge-mass-ratio}
\frac{e}{m_\text{e}}&=\frac{2\mu_\text{B}}{\hbar}.
\end{align}
So we finally get the value
\begin{align*}\frac{e}{m_\text{e}}&=\SI{1.84(15)e11}{\second\ampere\per\kilo\gram}
\end{align*}
This is in the 1$\sigma$-range of the literature value and only differs up to 4.6\% from this.\\
We conclude that it is possible to calculate the $\nicefrac{e}{m_\text{e}}$ ratio using the Zeeman effect, but for a better value with less error we would use an mass spectrometer.
For this, an electric field is combined with a magnetic field and the charge to mass ratio is obtained as a function of the radius in the form
\begin{align}
\frac{q}{m}=\frac{E}{B^2r}.
\end{align}

\begin{thebibliography}{}
\bibitem{}'Zeeman- und Paschen-Back-Effekt', \url{https://astro.uni-bonn.de/~joachimi/fp/Zeeman-Paschen-Back-Effekt.pdf}, 22.10.2018
\bibitem{}'Mass-to-Charge ratio', \url{https://en.wikipedia.org/wiki/Mass-to-charge_ratio}, 22.10.2018
\end{thebibliography}
\end{document}