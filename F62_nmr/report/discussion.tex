% !TeX root = f61_longreport_schmitt_kleinbek.tex
\section{Critical Comment}
This experiment was divided into work with a slightly older Bruker minispec p20 spectrometer and a Bruker NMR analyzer mq7.5.
First of all, it can be generally said that the older device reacts very sensitively to temperature fluctuations, which is why we often had to readjust the working frequency.
Nevertheless, the device was very descriptive and easily accessible from the outside, which makes it a good basis for nuclear magnetic resonance spectroscopy.\\
In the first part of the experiment we were able to measure the different relaxation times $T_1$ and $T_2$ of Gd500 and Gd600 with the device.
We found the theoretically expected differences between the times of a sample as well as between the two samples.
The times of Gd500 are smaller than those of Gd600 because Gadolinium facilitates alignment in the magnetic field.
The relaxation time $T_2$ of spin-spin interaction is less than that of spin-lattice $T_1$, because the latter includes the energy release into the environment, while the former only takes into account the behavior within the molecules.
In addition, the times of the spin echo method differ from those of the Carr-Purcell method because the spin-echo method gives the sample a certain time to diffuse and the magnetic field can also change during this time.\\
In the second section we have determined the chemical displacement of various samples.
This enabled us to assign functional groups and thus a chemical substance to each sample.
The samples were rotated to compensate for inhomogeneities in the magnetic field.
It is noticeable that the magnetic field is shielded differently by different groups, but the assignment by many overlaps is not so clear.
This would certainly not be possible without a specification of the possible substances.\\
Then, using the new NMR analyzer, we used nuclear magnetic resonance spectroscopy for imaging, where the intensity of individual pixels is proportional to the water content.
For example, we made one and two dimensional images of some organic and inorganic substances and examined their composition.
The correct placement in the analyzer was essential, which quickly became a bit time-consuming.
At the end, however, clear structures, such as the air pockets of a peanut, can be seen in the pictures.\\
Thus, the experiment was very good to gain insights into NMR, which is of course of great importance in medicine. 