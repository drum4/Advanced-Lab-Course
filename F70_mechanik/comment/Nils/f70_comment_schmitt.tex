\documentclass[12pt,
				 a4paper,
				%landscape,
				%headings=Big,
				numbers=endperiod
				 ]{scrartcl}
% !TeX root = f61_longreport_schmitt_kleinbek.tex
\usepackage{polyglossia}				%Sprache
\usepackage{amsmath}					%Matheumgebung
\usepackage{amsopn}					%Matheoperatoren
\usepackage{amssymb}					%Mathesymbole
\usepackage{amsthm}					%Mathetheorem
\usepackage{array}						%Tabellenumgebung
\usepackage[style=numeric,
			natbib=true,
			backend=biber]{biblatex}	%Bibliographie
\usepackage{blindtext}					%Blindtext
\usepackage{booktabs}					%Tabellen
\usepackage[style=english]{csquotes}	%Anführungszeichen
\usepackage{epstopdf}					%EPS Datei
\usepackage[includeheadfoot,
			top=1cm,
			bottom=1.5cm,
			left=2cm,
			right=2cm,
			]{geometry}
\usepackage{graphicx}					%Graphik
\usepackage{listings}					%Codeumgebung
\usepackage{mathdots}					%Punkte
\usepackage{mathtools}				%Bugfix ams
\usepackage{microtype}				%Makrotypographie
\usepackage{multicol}					%dreispaltig
\usepackage{multirow}					%mehrfache Zeilen
\usepackage{nicefrac}
\usepackage{pgfplots}					%Diagramme
\usepackage{physics}
\usepackage{relsize}					%Größenangaben
\usepackage{scrhack}
\usepackage[headsepline, automark]
			{scrlayer-scrpage}			%Kopfzeile
\usepackage[separate-uncertainty=true,
			quotient-mode=fraction,
			per-mode=symbol,
			list-final-separator={,},
			]{siunitx}					%Einheiten
\usepackage{subcaption}				%Gleitumgebung Graphik
\usepackage{tikz}						%Zeichnen
\usepackage{upgreek}					%Griechische Buchstaben
\usepackage{xcolor}					%Farben
\usepackage{xltxtra}					%fontec

\pgfplotsset{compat=1.15}
\setmainlanguage{english}
\setmainfont{Linux Libertine O}
\setsansfont{Linux Biolinum O}

\addbibresource{..\\..\\bib_fp.bib}

\ihead{F61 Nuclear magnetic resonance}
\ohead{\headmark}
\chead{}
\pagestyle{scrheadings}

\setcounter{tocdepth}{1}
\setcounter{secnumdepth}{1}
\setlength{\parindent}{0pt}

\usepackage[bookmarksopenlevel=section,
			linkcolor=blue,
			colorlinks=true,
			urlcolor=blue,
			citecolor=blue]
			{hyperref}		%Verweise, muss am Ende stehen

\titlehead{\begin{minipage}{.5\textwidth}
			\begin{flushleft}
			\includegraphics[scale=.065]{logo_color.jpg}
			\end{flushleft}
 	 		\end{minipage}
			\begin{minipage}{.5\textwidth}
			\begin{flushright}
			\includegraphics[scale=.8]{logo_fp.png}
			\end{flushright}
 	 		\end{minipage}}
\title{F70 Mechanik und Vakuum}
\subtitle{Kritische Würdigung}
\author{Nils Schmitt}
\publishers{Tutor: Bätzner, Rolf}
\date{Durchgeführt im September 2018}

\begin{document}
\maketitle
In diesem Versuch konnten wir die Arbeit eines Vakuumphysikers mittels zwei verschiedenen Vakuumumpen kennenlernen.\\

Dabei ließ sich an der Drehschieberpumpe der sogenannte Gasballast austesten um einen Basisdruck im Feinvakuumgebiet zu erreichen.
Mit der Wasserschale in der Vakuumkammer konnten wir so die Zustandsänderung von Wasser im Phasendiagramm nachvollziehen.
Auch wenn sich der Tripelpunkt leider nicht beobachten lies.\\
Der Basisdruck der Turbomolekularpumpe (TMP) ging bei uns bis unter \SI{e-6}{\milli\bar}.
Dies konnten wir aber nur über die lange Laufzeit erreichen, was jedoch schneller durch eine größere Öffnung zum Rezipienten gegangen wäre.
So konnten wir ein effektives Saugvermögen der TMP von $S_{\text{eff}}=\SI{65.6(11)}{\liter\per\second}$ bestimmen.
Der Fehler der Messung kommt hauptsächlich aus dem Zeitfehler, welcher also durch längere Teststrecken deutlich hätte verkleinert werden können.
Außerdem waren gerade die größeren Kapillare auch nicht ganz sauber zu bekommen, sodass der Durchmesser an manchen Stellen kleiner war.
Beim Kolbenmesser sollte dann noch angemerkt werden, dass wir den vom herrschenden Luftdruck sicher verschiedenen Normdruck verwendet haben, wourch auch hier eine Korrektur möglich ist.\\

Die anschließende Leitwertmessung von Rohr, Blende und der Kombination konnten wir die Kirchhoffschen Regeln ür die Reihenschaltung bestätigen.
Genauso weicht der Leitwert der Blende nicht signifikant von dem theoretischen Wert ab.
Der theoretische Leitwert des Rohs hingegen hängt bei viskoser Strömung vom Druck ab, während dies bei der Molekularströmung nicht der Fall ist.
In dem Bereich, in dem unsere Messung liegt, treten aber beide Strömungen auf, sodass unser Messwert zwischen den beiden erwarteten liegt.\\

Insgesamt muss man über den Versuch sagen, dass weit mehr als nur das Wissen über ein bisschen Vakuum vermittelt wurde.
So wurde die Abfrage auch durch viel Allgemeinwissen ergänzt und bei Umbauten konnten wir selbst Hand anlegen und etwas handwerkliche Fähigkeiten aufbauen. 
\end{document}